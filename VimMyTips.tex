\documentclass[10pt,landscape]{article}
\usepackage[top=2cm,bottom=2cm,left=2cm,right=2cm,a4paper]{geometry}
\usepackage{ucs}
\usepackage[utf8x]{inputenc}
\usepackage[T1]{fontenc}

\usepackage[protrusion=true,expansion=true]{microtype}
\usepackage{amssymb,amsmath}
\usepackage{eurosym}
\usepackage{fourier}
\usepackage{enumerate}
\usepackage{ctable}
\usepackage{float,lscape} % provides the H option for float placement
\usepackage{natbib}
\bibliographystyle{plainnat}
\usepackage{url}
\usepackage{longtable,ulem}

\usepackage{graphicx}
% We will generate all images so they have a width \maxwidth. This means
% that they will get their normal width if they fit onto the page, but
% are scaled down if they would overflow the margins.
\makeatletter
\def\maxwidth{\ifdim\Gin@nat@width>\linewidth0.7\linewidth
\else\Gin@nat@width\fi}
\makeatother
\let\Oldincludegraphics\includegraphics
\renewcommand{\includegraphics}[1]{\fbox{\Oldincludegraphics[width=\maxwidth]{#1}}}
\usepackage[unicode=true,
            colorlinks=true,
            linkcolor=blue]{hyperref}
\setlength{\headheight}{15pt}
\usepackage{xcolor,fancybox,fancyhdr}

\definecolor{bleufonce}{rgb}{0,0.4823529,0.549019}
\pagestyle{fancy}
\renewcommand\headrule{\color{bleufonce}\hrule height 2pt width\headwidth}
\pagenumbering{arabic}
\lhead{\colorbox{bleufonce}{\color{white}\textbf{Vim Memo}}}
\chead{}
\rhead{\itshape{\nouppercase{\leftmark}}}
\lfoot{}
\rfoot{}
\cfoot{}



% redefinir la taillle de l'environnement verbatim
\makeatletter
\def\verbatim{\footnotesize\@verbatim \frenchspacing\@vobeyspaces \@xverbatim}
\makeatother

\usepackage{fancyvrb}
\DefineShortVerb[commandchars=\\\{\}]{\|}
\DefineVerbatimEnvironment{Highlighting}{Verbatim}{fontsize=\footnotesize,commandchars=\\\{\}}
% Add ',fontsize=\small' for more characters per line
\newenvironment{Shaded}{}{}
\newcommand{\KeywordTok}[1]{\textcolor[rgb]{0.00,0.44,0.13}{\textbf{{#1}}}}
\newcommand{\DataTypeTok}[1]{\textcolor[rgb]{0.56,0.13,0.00}{{#1}}}
\newcommand{\DecValTok}[1]{\textcolor[rgb]{0.25,0.63,0.44}{{#1}}}
\newcommand{\BaseNTok}[1]{\textcolor[rgb]{0.25,0.63,0.44}{{#1}}}
\newcommand{\FloatTok}[1]{\textcolor[rgb]{0.25,0.63,0.44}{{#1}}}
\newcommand{\CharTok}[1]{\textcolor[rgb]{0.25,0.44,0.63}{{#1}}}
\newcommand{\StringTok}[1]{\textcolor[rgb]{0.25,0.44,0.63}{{#1}}}
\newcommand{\CommentTok}[1]{\textcolor[rgb]{0.38,0.63,0.69}{\textit{{#1}}}}
\newcommand{\OtherTok}[1]{\textcolor[rgb]{0.00,0.44,0.13}{{#1}}}
\newcommand{\AlertTok}[1]{\textcolor[rgb]{1.00,0.00,0.00}{\textbf{{#1}}}}
\newcommand{\FunctionTok}[1]{\textcolor[rgb]{0.02,0.16,0.49}{{#1}}}
\newcommand{\RegionMarkerTok}[1]{{#1}}
\newcommand{\ErrorTok}[1]{\textcolor[rgb]{1.00,0.00,0.00}{\textbf{{#1}}}}
\newcommand{\NormalTok}[1]{{#1}}
% Redefine labelwidth for lists; otherwise, the enumerate package will cause
% markers to extend beyond the left margin.
\makeatletter\AtBeginDocument{%
  \renewcommand{\@listi}
    {\setlength{\labelwidth}{4em}}
}\makeatother

\hypersetup{breaklinks=true, pdfborder={0 0 0}}
\setlength{\parindent}{0pt}
\setlength{\parskip}{6pt plus 2pt minus 1pt}
\setlength{\emergencystretch}{3em}  % prevent overfull lines
\widowpenalty=10000
\clubpenalty=10000
\usepackage{multicol}
\usepackage[frenchb]{babel}

\setcounter{secnumdepth}{0}
\VerbatimFootnotes % allows verbatim text in footnotes

\title{Vim Memo}
\author{NPO}
\date{Janvier 2014}

\begin{document}
%\maketitle

\begin{multicols}{2}
\begin{verbatim}
 |                            | MOVE TO...                                     |
 | :------------------------- | :--------------------------------------------- |
 | `h j k l`                  | Left, down, up, right                          |
 | `0 $ ^`                    | Start, end of line - first nonBlank char       |
 | `w e`                      | Start, end of word                             |
 | `gg G`                     | First, last line of file                       |
 | `Ctrl-d Ctrl-u`            | Half page down, up                             |
 | `Ctrl-b Ctrl-f`            | Page down, up                                  |

 |                            | SPLIT ET RESIZE                                |
 | :------------------------- | :--------------------------------------------- |
 | `:res 60`                  | Définit la hauteur de la fenêtre               |
 | `:vertical resize 80`      | Définit la largeur de la fenêtre               |
 | `:res {+,-}10`             | Augmente/réduit la hauteur de la fenêtre       |
 | `10 Ctrl-w {+,-}`          | "" idem                                        |
 | `:vertical resize {+,-}10` | Augmente/réduit la largeur de la fenêtre       |
 | `10 Ctrl-w {>,<}`          | " " idem                                       |
 | `Ctrl-w =`                 | Égalise la taille des fenêtres                 |
 | `Ctrl-w _`                 | Augmente au maximum la taille de la fenêtre    |
 | `Ctrl-w s`                 | Partage la fenêtre courante en deux            |
 | `Ctrl-w H`  `Ctrl-w L`     | Place la fenêtre courante à gauche /à droite   |
 | `Ctrl-w J`  `Ctrl-w K`     | Place la fenêtre courante en haut /en bas [^3] |

 |                            | BUFFERS  (better with bufferline)              |
 | :------------------------- | :--------------------------------------------- |
 | `:e newFile`               | Ouvre un nouveau buffer avec newFile           |
 | `:ls`                      | Pareil que :buffers, mais en plus court        |
 | `:bw`                      | Ferme le buffer courant                        |
 | `:sb x` `vsp | bx`         | Place le buffer x dans une fenêtre (v)splitée  |
 | `:sball`                   | Split all buffers                              |
 | `:bn :bp`   maped to F2 F3 | Opens next, previous buffer                    |

 |                            | SELECTION DE BLOC TEXTE                        |
 | :------------------------- | :--------------------------------------------- |
 | `v` `Ctrl-v`               | Selection de texte/ de colonne                 |
 | `V`                        | Selection de lignes entières                   |
 | `gv`                       | Resélectionne le bloc précédemment défini      |
 | `h j k l' , `D`            | *DRAGVISUAL*:    déplace bloc, duplique        |
 | `gq`                       | Formate la sélection                           |

 |                            | INSERTION SUR PLUSIEURS LIGNES                 |
 | :------------------------- | :--------------------------------------------- |
 | `Ctrl-v jj $ A foo Esc`    | Insertion à la fin des lignes d'un bloc        |
 | `Ctrl-v jj I foo Esc`      | Insertion dans une colonne d'un bloc           |
 | `Ctrl-v jj I # Esc`        | Commente les lignes d'un bloc                  |
 | `Shift-v jj s/^/#`         | Commente les lignes d'un bloc (autre méthode)  |




 |                            | RECHERCHE, REMPLACEMENT ET SUPPRESSIONS        |
 | :------------------------- | :--------------------------------------------- |
 | `*` `g*`                   | Recherche mot exact/partiel sous le curseur    |
 | `%`                        | Match brackets {}[]()                          |
 | `[I`                       | Affiche lignes contenant mot sous le curseur   |
 | `:g/foo`                   | Affiche les lignes contenant foo               |
 | `:g/foo/d`                 | Supprime les lignes contenant chaîne foo       |
 | `:v/foo/d`                 | Supprime lignes ne contenant pas foo           |
 | `:%s/\s\+$//`              | Supprime les espaces de fin de ligne           |
 | `:g/^[\.]*$/d`             | Supprime les lignes vides                      |
 | `:s/foo/bar/`              | Remplace le 1er  foo de ligne courante par bar |
 | `:s/.*\zsfoo/bar/`         | Remplace dernier foo de ligne courante par bar |
 | `:s/foo/bar/g`             | Remplace tous foo de ligne courante par bar    |
 | `:%s/foo/bar/g`            | Remplace TOUS foo du fichier par bar           |
 | `:%s/foo/bar/gc`           | " " idem avec demande de confirmation          |
 | `:%s/\<foo\>//g`           | Supprime le mot "foo".                         |
 | `:%s/.*\<foo\>//`          | Supprime "foo" et tout ce qui le précède.      |
 | `:%s/\<foo\>.*//`          | Supprime "foo" et tout ce qui le suit.         |
 | `:%s/.*\ze\<foo\>//`       | Supprime tout ce qui précède le mot "foo"      |
 | `:%s/\<foo\>\zs.*//`       | Supprime tout ce qui suit le mot "foo"         |
 | `:%s/\<foo\>.\{5}//`       | Supprime "foo" et les 5 catactères qui suivent |
 | `:%s#<\_.\{-1,}>##g        | Delete html tags possibly multi-line           |

       |           | ANCHORS                            |
       | `^`  `$`  | Start/end of line                  |
       | `\<` `\>` | Beginning/end of a word            |
       | `[ ]`     | Any characters listet              |
       | `[^ ]`    | Any characters except those listet |
       |                        | METACARACTÈRES                               |
       | `.` `\_.`              | Any character endOfLine (not) included       |
       | `\s`                   | Whitespace character                         |
       | `\d` `\a`              | Digit/alphabetic character                   |
       | `\w`                   | Word character [0-9A-Za-z_]                  |
       | `\l` `\u`              | Lowercase/uppercase  character               |
       | `\p` `\_p`             | Printable character endOfLine (not) included |
       | Greedy    : ^Greedy    | Quantifier                                   |
       | `* `      : `\{-} `    | 0 ou plus                                    |
       | `\+ `     : `\{-1} `   | 1 ou plus                                    |
       | `\= `     : `\{-0,1} ` | 0 ou 1 fois                                  |
       | `\{n} `   :            | n fois exactement                            |
       | `\{n,} `  : `\{-n,} `  | n fois au moins                              |
       | `\{,m} `  : `\{-,m} `  | m fois au plus                               |
       | `\{n,m} ` : `\{-n,m} ` | Entre n et m fois                            |
       |            | REPLACEMENT                                 |
       | `& `       | The whole matched pattern                   |
       | `\1 \2...` | Matches text in 1st, 2nd ... pair of `\(\)` |
       | `\l` `\u`  | Next char  made lowercase / uppercase       |
       | `\L` `\U`  | Next charS made lowercase / uppercase       | 
       | `\E`       | End of \u and \l                            |

 |                            | INDENTATION, AUTOINDENTATION ET *TABULARIZE*   |
 | :------------------------- | :--------------------------------------------- |
 | `>>` `<<`                  | Indente ou desindente la ligne courante
 | `gg=G`                     | Si marche pas, set `ft=html` + `set si`[^6]    |
 | `Shift-v =`                | Indente un bloc                                |
 | `==`                       | Indente la ligne courante                      |
 | `:Tabularize /:`           | Aligns statements on :                         |
 | `:Tabularize /&&`          | Aligns statements on &&                        |

 |                            | REGISTRES                                      |
 | :------------------------- | :--------------------------------------------- |
 | `"5yy"` `"hyy`             | Copie la ligne dans le registre 5 (ou h)       |
 | `:reg`                     | Liste les registres                            |
 | `"5p`  `"hp`               | Colle le contenu du regsitre 5 (ou h)          |
 | `"_dd`                     | Delete to BlackHole (don't affect any register |

 |                            | FOLDING COMMANDS [^1]                          |
 | :------------------------- | :--------------------------------------------- |
 | `za`                       | Toggle state of one fold (`zA` recurs [^2])    |
 | `zR`                       | Opens ALL folds (`zr` decr foldlevel by 1)     |
 | `zM`                       | Closes ALL folds (`zM` incr foldlevel by 1)    |
 | `zf%`                      | Creates fold from a delimitor to its brother   |
 | `V jj zf`                  | Creates fold from visual block                 |
 | `zj` `zk`                  | Move to the next, previous fold                |
 | `[z` `]z`                  | Move to start/ end of open fold                |

 |                            | RECORDING                                      |
 | :------------------------- | :--------------------------------------------- |
 | `qa` ... `qz`              | Démarre enregistrement action (registre a...z) |
 | `q`                        | Stopppe enregistrement action [^8]             |
 | `@a` `@z`                  | Joue  l'enregistrement action (registre a...z) |
 | `6@a`                      | Joue 6x enregistrement action                  |
 | `@@`                       | Rejoue le dernier enregistrement               |
 | `6@@`                      | Rejoue 6x le dernier enregistrement            |
 | `"ap` `<C-R>a              | Affiche enregistrement normal/insert mode      |
 | `"add`                     | Réenregistre action dans le registre a         |



 |                            | DIVERS                                         |
 | :------------------------- | :--------------------------------------------- |
 | `.`                        | Repeat last modification                       |
 | `@:`                       | Repeat last : command (then @@)                |
 | `:&` ou `:~`               | Last substitute                                |
 | `g%`                       | Normal mode repeat last substitute             |
 | `:history`                 | List of all your commands                      |
 | `:Sex(plore)`              | File explorer in split window                  |
 | `<C-X><C-L>`               | Line complete SUPER USEFUL                     |
 | `/<C-R><C-W>`              | Pull <cword> onto search/command line          |
 | `:digraphs` `ga`           | Display table / hex value of char under cursor |
 | `i<C-V>233`                | Insert é                                       |
 | `<C-A>,<C-X>`              | Increment,decrement number under cursor        |
 | `:set paste`               | Avant de coller depuis le navigateur           |
 | `:set nopaste`             | Après avoir collé depuis le navigateur [^7]    |
 | `:sort [n]`                | Trier sur première colonne [numérique] [^10]   |
 | `:%!sort -n -k 3           | À l'aide de gnu sort                           |

 |                            | PLUGINS                                        |
 | :------------------------- | :--------------------------------------------- |
 | `<leader><leader>w`        | *EASYMOTION*                                   |
 | `<leader>gs`               | *FUGITIVE* :Gstatus `-`un/stage, `cc`commt msg |
 | `:TagbarToggle` maped to F8| *TAGBAR*                                       |

 |                            | TAGS and *SURROUND*                            |
 | :------------------------- | :--------------------------------------------- |
 | `S<h3>`                    | Add Surround to visual sélection               |
 | `{d,y,v}it`                | Deletes, yanks or visual Inside Tags           |
 | `cst<h3>`                  | Change Surrounding Tag (current) to <h3>       |
 | `cs">` `cs"<h3>`           | Change Surrounding " to  <..> / <h3></h3>      |
 | `ds"`                      | Delete le surround "                           |

[^1]:  Dans le .vimrc `:mkview    " save folds` & `:lowercaseadview  " restore folds`
[^2]: za (resp. zA) toggles between zo & zc (resp. zO & zC)
[^3]: The lower case equivalents move focus instead of moving the window.
[^6]: donner le bon le filetype et enclancher le smartindent
[^7]: More info in http://www.vim.org/tips/tip.php?tip_id=330 
           *Autre méthode* :set pastetoggle=<F3>

\end{verbatim}
\end{multicols}

\newpage

\begin{multicols}{2}
\begin{verbatim}

__BEGIN__
*vimtips.txt*	For Vim version 7.3.  
------------------------------------------------------------------------------
" new items marked [N] , corrected items marked [C]
" *best-searching*
/joe/e                      : cursor set to End of match
3/joe/e+1                   : find 3rd joe cursor set to End of match plus 1 [C]
/joe/s-2                    : cursor set to Start of match minus 2
/joe/+3                     : find joe move cursor 3 lines down
/^joe.*fred.*bill/          : find joe AND fred AND Bill (Joe at start of line)
/^[A-J]/                    : search for lines beginning with one or more A-J
/begin\_.*end               : search over possible multiple lines
/fred\_s*joe/               : any whitespace including newline [C]
/fred\|joe                  : Search for FRED OR JOE
/.*fred\&.*joe              : Search for FRED AND JOE in any ORDER!
/\<fred\>/                  : search for fred but not alfred or frederick [C]
/\<\d\d\d\d\>               : Search for exactly 4 digit numbers
/\D\d\d\d\d\D               : Search for exactly 4 digit numbers
/\<\d\{4}\>                 : same thing
/\([^0-9]\|^\)%.*%          : Search for absence of a digit or beginning of line
" finding empty lines
/^\n\{3}                    : find 3 empty lines
/^str.*\nstr                : find 2 successive lines starting with str
/\(^str.*\n\)\{2}           : find 2 successive lines starting with str
" using rexexp memory in a search find fred.*joe.*joe.*fred *C*
/\(fred\).*\(joe\).*\2.*\1
" Repeating the Regexp (rather than what the Regexp finds)
/^\([^,]*,\)\{8}
" visual searching
:vmap // y/<C-R>"<CR>       : search for visually highlighted text
:vmap <silent> //    y/<C-R>=escape(@", '\\/.*$^~[]')<CR><CR> : with spec chars
" \zs and \ze regex delimiters :h /\zs
/<\zs[^>]*\ze>              : search for tag contents, ignoring chevrons
" zero-width :h /\@=
/<\@<=[^>]*>\@=             : search for tag contents, ignoring chevrons
/<\@<=\_[^>]*>\@=           : search for tags across possible multiple lines
" searching over multiple lines \_ means including newline
/<!--\_p\{-}-->                   : search for multiple line comments
/fred\_s*joe/                     : any whitespace including newline *C*
/bugs\(\_.\)*bunny                : bugs followed by bunny anywhere in file
:h \_                             : help
" search for declaration of subroutine/function under cursor
:nmap gx yiw/^\(sub\<bar>function\)\s\+<C-R>"<CR>
" multiple file search
:bufdo /searchstr/                : use :rewind to recommence search
" multiple file search better but cheating
:bufdo %s/searchstr/&/gic   : say n and then a to stop
" How to search for a URL without backslashing
?http://www.vim.org/        : (first) search BACKWARDS!!! clever huh!
" Specify what you are NOT searching for (vowels)
/\c\v([^aeiou]&\a){4}       : search for 4 consecutive consonants
/\%>20l\%<30lgoat           : Search for goat between lines 20 and 30 [N]
/^.\{-}home.\{-}\zshome/e   : match only the 2nd occurence in a line of "home" [N]
:%s/home.\{-}\zshome/alone  : Substitute only the occurrence of home in any line [N]
" find str but not on lines containing tongue
^\(.*tongue.*\)\@!.*nose.*$
\v^((tongue)@!.)*nose((tongue)@!.)*$
.*nose.*\&^\%(\%(tongue\)\@!.\)*$ 
:v/tongue/s/nose/&/gic
'a,'bs/extrascost//gc       : trick: restrict search to between markers (answer n) [N]
"----------------------------------------
" *best-substitution*
:%s/fred/joe/igc            : general substitute command
:%s//joe/igc                : Substitute what you last searched for [N]
:%s/~/sue/igc               : Substitute your last replacement string [N]
:%s/\r//g                   : Delete DOS returns ^M
" Is your Text File jumbled onto one line? use following
:%s/\r/\r/g                 : Turn DOS returns ^M into real returns
:%s=  *$==                  : delete end of line blanks
:%s= \+$==                  : Same thing
:%s#\s*\r\?$##              : Clean both trailing spaces AND DOS returns
:%s#\s*\r*$##               : same thing
" deleting empty lines
:%s/^\n\{3}//               : delete blocks of 3 empty lines
:%s/^\n\+/\r/               : compressing empty lines
:%s#<[^>]\+>##g             : delete html tags, leave text (non-greedy)
:%s#<\_.\{-1,}>##g          : delete html tags possibly multi-line (non-greedy)
:%s#.*\(\d\+hours\).*#\1#   : Delete all but memorised string (\1) [N]
" parse xml/soap 
%s#><\([^/]\)#>\r<\1#g      : split jumbled up XML file into one tag per line [N]
%s/</\r&/g                  : simple split of html/xml/soap  [N]
:%s#<[^/]#\r&#gic           : simple split of html/xml/soap  but not closing tag [N]
:%s#<[^/]#\r&#gi            : parse on open xml tag [N]
:%s#\[\d\+\]#\r&#g          : parse on numbered array elements [1] [N]
" VIM Power Substitute
:'a,'bg/fred/s/dick/joe/igc : VERY USEFUL
" duplicating columns
:%s= [^ ]\+$=&&=            : duplicate end column
:%s= \f\+$=&&=              : Dupicate filename
:%s= \S\+$=&&               : usually the same
" memory
:%s#example#& = &#gic        : duplicate entire matched string [N]
:%s#.*\(tbl_\w\+\).*#\1#    : extract list of all strings tbl_* from text  [NC]
:s/\(.*\):\(.*\)/\2 : \1/   : reverse fields separated by :
:%s/^\(.*\)\n\1$/\1/        : delete duplicate lines
:%s/^\(.*\)\(\n\1\)\+$/\1/  : delete multiple duplicate lines [N]
" non-greedy matching \{-}
:%s/^.\{-}pdf/new.pdf/      : delete to 1st occurence of pdf only (non-greedy)
%s#^.\{-}\([0-9]\{3,4\}serial\)#\1#gic : delete up to 123serial or 1234serial [N]
" use of optional atom \?
:%s#\<[zy]\?tbl_[a-z_]\+\>#\L&#gc : lowercase with optional leading characters
" over possibly many lines
:%s/<!--\_.\{-}-->//        : delete possibly multi-line comments
:help /\{-}                 : help non-greedy
" substitute using a register
:s/fred/<c-r>a/g            : sub "fred" with contents of register "a"
:s/fred/<c-r>asome_text<c-r>s/g  
:s/fred/\=@a/g              : better alternative as register not displayed (not *) [C]
" multiple commands on one line
:%s/\f\+\.gif\>/\r&\r/g | v/\.gif$/d | %s/gif/jpg/
:%s/a/but/gie|:update|:next : then use @: to repeat
" ORing
:%s/goat\|cow/sheep/gc      : ORing (must break pipe)
:'a,'bs#\[\|\]##g           : remove [] from lines between markers a and b [N]
:%s/\v(.*\n){5}/&\r         : insert a blank line every 5 lines [N]
" Calling a VIM function
:s/__date__/\=strftime("%c")/ : insert datestring
:inoremap \zd <C-R>=strftime("%d%b%y")<CR>    : insert date eg 31Jan11 [N]
" Working with Columns sub any str1 in col3
:%s:\(\(\w\+\s\+\)\{2}\)str1:\1str2:
" Swapping first & last column (4 columns)
:%s:\(\w\+\)\(.*\s\+\)\(\w\+\)$:\3\2\1:
" format a mysql query 
:%s#\<from\>\|\<where\>\|\<left join\>\|\<\inner join\>#\r&#g
" filter all form elements into paste register
:redir @*|sil exec 'g#<\(input\|select\|textarea\|/\=form\)\>#p'|redir END
:nmap ,z :redir @*<Bar>sil exec 'g@<\(input\<Bar>select\<Bar>textarea\<Bar>/\=form\)\>@p'<Bar>redir END<CR>
" substitute string in column 30 [N]
:%s/^\(.\{30\}\)xx/\1yy/
" decrement numbers by 3
:%s/\d\+/\=(submatch(0)-3)/
" increment numbers by 6 on certain lines only
:g/loc\|function/s/\d/\=submatch(0)+6/
" better
:%s#txtdev\zs\d#\=submatch(0)+1#g
:h /\zs
" increment only numbers gg\d\d  by 6 (another way)
:%s/\(gg\)\@<=\d\+/\=submatch(0)+6/
:h zero-width
" rename a string with an incrementing number
:let i=10 | 'a,'bg/Abc/s/yy/\=i/ |let i=i+1 # convert yy to 10,11,12 etc
" as above but more precise
:let i=10 | 'a,'bg/Abc/s/xx\zsyy\ze/\=i/ |let i=i+1 # convert xxyy to xx11,xx12,xx13
" find replacement text, put in memory, then use \zs to simplify substitute
:%s/"\([^.]\+\).*\zsxx/\1/
" Pull word under cursor into LHS of a substitute
:nmap <leader>z :%s#\<<c-r>=expand("<cword>")<cr>\>#
" Pull Visually Highlighted text into LHS of a substitute
:vmap <leader>z :<C-U>%s/\<<c-r>*\>/
" substitute singular or plural
:'a,'bs/bucket\(s\)*/bowl\1/gic   [N]
----------------------------------------
" all following performing similar task, substitute within substitution
" Multiple single character substitution in a portion of line only
:%s,\(all/.*\)\@<=/,_,g     : replace all / with _ AFTER "all/"
" Same thing
:s#all/\zs.*#\=substitute(submatch(0), '/', '_', 'g')#
" Substitute by splitting line, then re-joining
:s#all/#&^M#|s#/#_#g|-j!
" Substitute inside substitute
:%s/.*/\='cp '.submatch(0).' all/'.substitute(submatch(0),'/','_','g')/
----------------------------------------
" *best-global* command 
:g/gladiolli/#              : display with line numbers (YOU WANT THIS!)
:g/fred.*joe.*dick/         : display all lines fred,joe & dick
:g/\<fred\>/                : display all lines fred but not freddy
:g/^\s*$/d                  : delete all blank lines
:g!/^dd/d                   : delete lines not containing string
:v/^dd/d                    : delete lines not containing string
:g/joe/,/fred/d             : not line based (very powerfull)
:g/fred/,/joe/j             : Join Lines [N]
:g/-------/.-10,.d          : Delete string & 10 previous lines
:g/{/ ,/}/- s/\n\+/\r/g     : Delete empty lines but only between {...}
:v/\S/d                     : Delete empty lines (and blank lines ie whitespace)
:v/./,/./-j                 : compress empty lines
:g/^$/,/./-j                : compress empty lines
:g/<input\|<form/p          : ORing
:g/^/put_                   : double space file (pu = put)
:g/^/m0                     : Reverse file (m = move)
:g/^/m$                     : No effect! [N]
:'a,'bg/^/m'b               : Reverse a section a to b
:g/^/t.                     : duplicate every line
:g/fred/t$                  : copy (transfer) lines matching fred to EOF
:g/stage/t'a                : copy (transfer) lines matching stage to marker a (cannot use .) [C]
:g/^Chapter/t.|s/./-/g      : Automatically underline selecting headings [N]
:g/\(^I[^^I]*\)\{80}/d      : delete all lines containing at least 80 tabs
" perform a substitute on every other line
:g/^/ if line('.')%2|s/^/zz / 
" match all lines containing "somestr" between markers a & b
" copy after line containing "otherstr"
:'a,'bg/somestr/co/otherstr/ : co(py) or mo(ve)
" as above but also do a substitution
:'a,'bg/str1/s/str1/&&&/|mo/str2/
:%norm jdd                  : delete every other line
" incrementing numbers (type <c-a> as 5 characters)
:.,$g/^\d/exe "norm! \<c-a>": increment numbers
:'a,'bg/\d\+/norm! ^A       : increment numbers
" storing glob results (note must use APPEND) you need to empty reg a first with qaq. 
"save results to a register/paste buffer
:g/fred/y A                 : append all lines fred to register a
:g/fred/y A | :let @*=@a    : put into paste buffer
:let @a=''|g/Barratt/y A |:let @*=@a
" filter lines to a file (file must already exist)
:'a,'bg/^Error/ . w >> errors.txt
" duplicate every line in a file wrap a print '' around each duplicate
:g/./yank|put|-1s/'/"/g|s/.*/Print '&'/
" replace string with contents of a file, -d deletes the "mark"
:g/^MARK$/r tmp.txt | -d
" display prettily
:g/<pattern>/z#.5           : display with context
:g/<pattern>/z#.5|echo "=========="  : display beautifully
" Combining g// with normal mode commands
:g/|/norm 2f|r*                      : replace 2nd | with a star
"send output of previous global command to a new window
:nmap <F3>  :redir @a<CR>:g//<CR>:redir END<CR>:new<CR>:put! a<CR><CR>
"----------------------------------------
" *Best-Global-combined-with-substitute* (*power-editing*)
:'a,'bg/fred/s/joe/susan/gic :  can use memory to extend matching
:/fred/,/joe/s/fred/joe/gic :  non-line based (ultra)
:/biz/,/any/g/article/s/wheel/bucket/gic:  non-line based [N]
----------------------------------------
" Find fred before beginning search for joe
:/fred/;/joe/-2,/sid/+3s/sally/alley/gIC
"----------------------------------------
" create a new file for each line of file eg 1.txt,2.txt,3,txt etc
:g/^/exe ".w ".line(".").".txt"
"----------------------------------------
" chain an external command
:.g/^/ exe ".!sed 's/N/X/'" | s/I/Q/    [N]
"----------------------------------------
" Operate until string found [N]
d/fred/                                :delete until fred
y/fred/                                :yank until fred
c/fred/e                               :change until fred end
v12|                                   : visualise/change/delete to column 12 [N]
"----------------------------------------
" Summary of editing repeats [N]
.      last edit (magic dot)
:&     last substitute
:%&    last substitute every line
:%&gic last substitute every line confirm
g%     normal mode repeat last substitute
g&     last substitute on all lines
@@     last recording
@:     last command-mode command
:!!    last :! command
:~     last substitute
:help repeating
----------------------------------------
" Summary of repeated searches
;      last f, t, F or T
,      last f, t, F or T in opposite direction
n      last / or ? search
N      last / or ? search in opposite direction
----------------------------------------
" *Absolutely-essential*
----------------------------------------
* # g* g#           : find word under cursor (<cword>) (forwards/backwards)
%                   : match brackets {}[]()
.                   : repeat last modification 
@:                  : repeat last : command (then @@)
matchit.vim         : % now matches tags <tr><td><script> <?php etc
<C-N><C-P>          : word completion in insert mode
<C-X><C-L>          : Line complete SUPER USEFUL
/<C-R><C-W>         : Pull <cword> onto search/command line
/<C-R><C-A>         : Pull <CWORD> onto search/command line
:set ignorecase     : you nearly always want this
:set smartcase      : overrides ignorecase if uppercase used in search string (cool)
:syntax on          : colour syntax in Perl,HTML,PHP etc
:set syntax=perl    : force syntax (usually taken from file extension)
:h regexp<C-D>      : type control-D and get a list all help topics containing
                      regexp (plus use TAB to Step thru list)
----------------------------------------
" MAKE IT EASY TO UPDATE/RELOAD _vimrc
:nmap ,s :source $VIM/_vimrc
:nmap ,v :e $VIM/_vimrc
:e $MYVIMRC         : edits your _vimrc whereever it might be  [N]
" How to have a variant in your .vimrc for different PCs [N]
if $COMPUTERNAME == "NEWPC"
ab mypc vista
else
ab mypc dell25
endif
----------------------------------------
" splitting windows
:vsplit other.php       # vertically split current file with other.php [N]
----------------------------------------
"VISUAL MODE (easy to add other HTML Tags)
:vmap sb "zdi<b><C-R>z</b><ESC>  : wrap <b></b> around VISUALLY selected Text
:vmap st "zdi<?= <C-R>z ?><ESC>  : wrap <?=   ?> around VISUALLY selected Text
----------------------------------------
"vim 7 tabs
vim -p fred.php joe.php             : open files in tabs
:tabe fred.php                      : open fred.php in a new tab
:tab ball                           : tab open files
:close                              : close a tab but leave the buffer *N*
" vim 7 forcing use of tabs from .vimrc
:nnoremap gf <C-W>gf
:cab      e  tabe
:tab sball                           : retab all files in buffer (repair) [N]
----------------------------------------
" Exploring
:e .                            : file explorer
:Exp(lore)                      : file explorer note capital Ex
:Sex(plore)                     : file explorer in split window
:browse e                       : windows style browser
:ls                             : list of buffers
:cd ..                          : move to parent directory
:args                           : list of files
:pwd                            : Print Working Directory (current directory) [N]
:args *.php                     : open list of files (you need this!)
:lcd %:p:h                      : change to directory of current file
:autocmd BufEnter * lcd %:p:h   : change to directory of current file automatically (put in _vimrc)
----------------------------------------
" Changing Case
guu                             : lowercase line
gUU                             : uppercase line
Vu                              : lowercase line
VU                              : uppercase line
g~~                             : flip case line
vEU                             : Upper Case Word
vE~                             : Flip Case Word
ggguG                           : lowercase entire file
" Titlise Visually Selected Text (map for .vimrc)
vmap ,c :s/\<\(.\)\(\k*\)\>/\u\1\L\2/g<CR>
" Title Case A Line Or Selection (better)
vnoremap <F6> :s/\%V\<\(\w\)\(\w*\)\>/\u\1\L\2/ge<cr> [N]
" titlise a line
nmap ,t :s/.*/\L&/<bar>:s/\<./\u&/g<cr>  [N]
" Uppercase first letter of sentences
:%s/[.!?]\_s\+\a/\U&\E/g
----------------------------------------
gf                              : open file name under cursor (SUPER)
:nnoremap gF :view <cfile><cr>  : open file under cursor, create if necessary
ga                              : display hex,ascii value of char under cursor
ggVGg?                          : rot13 whole file
ggg?G                           : rot13 whole file (quicker for large file)
:8 | normal VGg?                : rot13 from line 8
:normal 10GVGg?                 : rot13 from line 8
<C-A>,<C-X>                     : increment,decrement number under cursor
                                  win32 users must remap CNTRL-A
<C-R>=5*5                       : insert 25 into text (mini-calculator)
----------------------------------------
" Make all other tips superfluous
:h 42            : also http://www.google.com/search?q=42
:h holy-grail
:h!
----------------------------------------
" disguise text (watch out) [N]
ggVGg?                          : rot13 whole file (toggles)
:set rl!                        : reverse lines right to left (toggles)
:g/^/m0                         : reverse lines top to bottom (toggles)
:%s/\(\<.\{-}\>\)/\=join(reverse(split(submatch(1), '.\zs')), '')/g   : reverse all text *N*
----------------------------------------
" History, Markers & moving about (what Vim Remembers) [C]
'.               : jump to last modification line (SUPER)
`.               : jump to exact spot in last modification line
g;               : cycle thru recent changes (oldest first)
g,               : reverse direction 
:changes
:h changelist    : help for above
<C-O>            : retrace your movements in file (starting from most recent)
<C-I>            : retrace your movements in file (reverse direction)
:ju(mps)         : list of your movements
:help jump-motions
:history         : list of all your commands
:his c           : commandline history
:his s           : search history
q/               : Search history Window (puts you in full edit mode) (exit CTRL-C)
q:               : commandline history Window (puts you in full edit mode) (exit CTRL-C)
:<C-F>           : history Window (exit CTRL-C)
----------------------------------------
" Abbreviations & Maps
" Maps are commands put onto keys, abbreviations expand typed text [N]
" Following 4 maps enable text transfer between VIM sessions
:map   <f7>   :'a,'bw! c:/aaa/x       : save text to file x
:map   <f8>   :r c:/aaa/x             : retrieve text 
:map   <f11>  :.w! c:/aaa/xr<CR>      : store current line
:map   <f12>  :r c:/aaa/xr<CR>        : retrieve current line
:ab php          : list of abbreviations beginning php
:map ,           : list of maps beginning ,
" allow use of F10 for mapping (win32)
set wak=no       : :h winaltkeys
" For use in Maps
<CR>             : carriage Return for maps
<ESC>            : Escape
<LEADER>         : normally \
<BAR>            : | pipe
<BACKSPACE>      : backspace
<SILENT>         : No hanging shell window
"display RGB colour under the cursor eg #445588
:nmap <leader>c :hi Normal guibg=#<c-r>=expand("<cword>")<cr><cr>
map <f2> /price only\\|versus/ :in a map need to backslash the \
" type table,,, to get <table></table>       ### Cool ###
imap ,,, <esc>bdwa<<esc>pa><cr></<esc>pa><esc>kA
" list current mappings of all your function keys
:for i in range(1, 12) | execute("map <F".i.">") | endfor   [N]
" for your .vimrc
:cab ,f :for i in range(1, 12) \| execute("map <F".i.">") \| endfor
"chain commands in abbreviation
cabbrev vrep tabe class.inc \| tabe report.php   ## chain commands [N]
----------------------------------------
" Simple PHP debugging display all variables yanked into register a
iab phpdb exit("<hr>Debug <C-R>a  ");
----------------------------------------
" Using a register as a map (preload registers in .vimrc)
:let @m=":'a,'bs/"
:let @s=":%!sort -u"
----------------------------------------
" Useful tricks
"ayy@a           : execute "Vim command" in a text file
yy@"             : same thing using unnamed register
u@.              : execute command JUST typed in
"ddw             : store what you delete in register d [N]
"ccaw            : store what you change in register c [N]
----------------------------------------
" Get output from other commands (requires external programs)
:r!ls -R         : reads in output of ls
:put=glob('**')  : same as above                 [N]
:r !grep "^ebay" file.txt  : grepping in content   [N]
:20,25 !rot13    : rot13 lines 20 to 25   [N]
!!date           : same thing (but replaces/filters current line)
" Sorting with external sort
:%!sort -u       : use an external program to filter content
:'a,'b!sort -u   : use an external program to filter content
!1} sort -u      : sorts paragraph (note normal mode!!)
:g/^$/;/^$/-1!sort : Sort each block (note the crucial ;)
" Sorting with internal sort
:sort /.*\%2v/   : sort all lines on second column [N]
" number lines  (linux or cygwin only)
:new | r!nl #                  [N]
----------------------------------------
" Multiple Files Management (Essential)
:bn              : goto next buffer
:bp              : goto previous buffer
:wn              : save file and move to next (super)
:wp              : save file and move to previous
:bd              : remove file from buffer list (super)
:bun             : Buffer unload (remove window but not from list)
:badd file.c     : file from buffer list
:b3              : go to buffer 3 [C]
:b main          : go to buffer with main in name eg main.c (ultra)
:sav php.html    : Save current file as php.html and "move" to php.html
:sav! %<.bak     : Save Current file to alternative extension (old way)
:sav! %:r.cfm    : Save Current file to alternative extension
:sav %:s/fred/joe/           : do a substitute on file name
:sav %:s/fred/joe/:r.bak2    : do a substitute on file name & ext.
:!mv % %:r.bak   : rename current file (DOS use Rename or DEL)
:help filename-modifiers
:e!              : return to unmodified file
:w c:/aaa/%      : save file elsewhere
:e #             : edit alternative file (also cntrl-^)
:rew             : return to beginning of edited files list (:args)
:brew            : buffer rewind
:sp fred.txt     : open fred.txt into a split
:sball,:sb       : Split all buffers (super)
:scrollbind      : in each split window
:map   <F5> :ls<CR>:e # : Pressing F5 lists all buffer, just type number
:set hidden      : Allows to change buffer w/o saving current buffer
----------------------------------------
" Quick jumping between splits
map <C-J> <C-W>j<C-W>_
map <C-K> <C-W>k<C-W>_
----------------------------------------
" Recording (BEST Feature of ALL)
qq  # record to q
your complex series of commands
q   # end recording
@q to execute
@@ to Repeat
5@@ to Repeat 5 times
qQ@qq                             : Make an existing recording q recursive [N]
" editing a register/recording
"qp                               :display contents of register q (normal mode)
<ctrl-R>q                         :display contents of register q (insert mode)
" you can now see recording contents, edit as required
"qdd                              :put changed contacts back into q
@q                                :execute recording/register q
" Operating a Recording on a Visual BLOCK (blockwise)
1) define recording/register
qq:s/ to/ from/g^Mq
2) Define Visual BLOCK
V}
3) hit : and the following appears
:'<,'>
4)Complete as follows
:'<,'>norm @q
----------------------------------------
"combining a recording with a map (to end up in command mode)
"here we operate on a file with a recording, then move to the next file [N]
:nnoremap ] @q:update<bar>bd
----------------------------------------
" Visual is the newest and usually the most intuitive editing mode
" Visual basics
v                               : enter visual mode
V                               : visual mode whole line
<C-V>                           : enter VISUAL BLOCKWISE mode (remap on Windows to say C-Q *C*
gv                              : reselect last visual area (ultra)
o                               : navigate visual area
"*y or "+y                      : yank visual area into paste buffer  [C]
V%                              : visualise what you match
V}J                             : Join Visual block (great)
V}gJ                            : Join Visual block w/o adding spaces
`[v`]                           : Highlight last insert
:%s/\%Vold/new/g                : Do a substitute on last visual area [N]
----------------------------------------
" Delete 8th and 9th characters of 10 successive lines [C]
08l<c-v>10j2ld  (use Control Q on win32) [C]
----------------------------------------
" how to copy a set of columns using VISUAL BLOCK
" visual block (AKA columnwise selection) (NOT BY ordinary v command)
<C-V> then select "column(s)" with motion commands (win32 <C-Q>)
then c,d,y,r etc
----------------------------------------
" how to overwrite a visual-block of text with another such block [C]
" move with hjkl etc
Pick the first block: ctrl-v move y
Pick the second block: ctrl-v move P <esc>
----------------------------------------
" text objects :h text-objects                                     [C]
daW                                   : delete contiguous non whitespace
di<   yi<  ci<                        : Delete/Yank/Change HTML tag contents
da<   ya<  ca<                        : Delete/Yank/Change whole HTML tag
dat   dit                             : Delete HTML tag pair
diB   daB                             : Empty a function {}
das                                   : delete a sentence
----------------------------------------
" _vimrc essentials
:imap <TAB> <C-N>                     : set tab to complete [N]
:set incsearch : jumps to search word as you type (annoying but excellent)
:set wildignore=*.o,*.obj,*.bak,*.exe : tab complete now ignores these
:set shiftwidth=3                     : for shift/tabbing
:set vb t_vb=".                       : set silent (no beep)
:set browsedir=buffer                 : Maki GUI File Open use current directory
----------------------------------------
" launching Win IE
:nmap ,f :update<CR>:silent !start c:\progra~1\intern~1\iexplore.exe file://%:p<CR>
:nmap ,i :update<CR>: !start c:\progra~1\intern~1\iexplore.exe <cWORD><CR>
----------------------------------------
" FTPing from VIM
cmap ,r  :Nread ftp://209.51.134.122/public_html/index.html
cmap ,w  :Nwrite ftp://209.51.134.122/public_html/index.html
gvim ftp://www.somedomain.com/index.html # uses netrw.vim
----------------------------------------
" appending to registers (use CAPITAL)
" yank 5 lines into "a" then add a further 5
"a5yy
10j
"A5yy
----------------------------------------
[I     : show lines matching word under cursor <cword> (super)
----------------------------------------
" Conventional Shifting/Indenting
:'a,'b>>
" visual shifting (builtin-repeat)
:vnoremap < <gv
:vnoremap > >gv
" Block shifting (magic)
>i{
>a{
" also
>% and <%
==                            : index current line same as line above [N]
----------------------------------------
" Redirection & Paste register *
:redir @*                    : redirect commands to paste buffer
:redir END                   : end redirect
:redir >> out.txt            : redirect to a file
" Working with Paste buffer
"*yy                         : yank curent line to paste
"*p                          : insert from paste buffer
" yank to paste buffer (ex mode)
:'a,'by*                     : Yank range into paste
:%y*                         : Yank whole buffer into paste
:.y*                         : Yank Current line to paster
" filter non-printable characters from the paste buffer
" useful when pasting from some gui application
:nmap <leader>p :let @* = substitute(@*,'[^[:print:]]','','g')<cr>"*p
:set paste                    : prevent vim from formatting pasted in text *N*
----------------------------------------
" Re-Formatting text
gq}                          : Format a paragraph
gqap                         : Format a paragraph
ggVGgq                       : Reformat entire file
Vgq                          : current line
" break lines at 70 chars, if possible after a ;
:s/.\{,69\};\s*\|.\{,69\}\s\+/&\r/g
----------------------------------------
" Operate command over multiple files
:argdo %s/foo/bar/e          : operate on all files in :args
:bufdo %s/foo/bar/e
:windo %s/foo/bar/e
:argdo exe '%!sort'|w!       : include an external command
:bufdo exe "normal @q" | w   : perform a recording on open files
:silent bufdo !zip proj.zip %:p   : zip all current files
----------------------------------------
" Command line tricks
gvim -h                    : help
ls | gvim -                : edit a stream!!
cat xx | gvim - -c "v/^\d\d\|^[3-9]/d " : filter a stream
gvim -o file1 file2        : open into a horizontal split (file1 on top, file2 on bottom) [C]
gvim -O file1 file2        : open into a vertical split (side by side,for comparing code) [N]
" execute one command after opening file
gvim.exe -c "/main" joe.c  : Open joe.c & jump to "main"
" execute multiple command on a single file
vim -c "%s/ABC/DEF/ge | update" file1.c
" execute multiple command on a group of files
vim -c "argdo %s/ABC/DEF/ge | update" *.c
" remove blocks of text from a series of files
vim -c "argdo /begin/+1,/end/-1g/^/d | update" *.c
" Automate editing of a file (Ex commands in convert.vim)
vim -s "convert.vim" file.c
"load VIM without .vimrc and plugins (clean VIM) e.g. for HUGE files
gvim -u NONE -U NONE -N
" Access paste buffer contents (put in a script/batch file)
gvim -c 'normal ggdG"*p' c:/aaa/xp
" print paste contents to default printer
gvim -c 's/^/\=@*/|hardcopy!|q!'
" gvim's use of external grep (win32 or *nix)
:!grep somestring *.php     : creates a list of all matching files [C]
" use :cn(ext) :cp(rev) to navigate list
:h grep
" Using vimgrep with copen                              [N]
:vimgrep /keywords/ *.php
:copen
----------------------------------------
" GVIM Difference Function (Brilliant)
gvim -d file1 file2        : vimdiff (compare differences)
dp                         : "put" difference under cursor to other file
do                         : "get" difference under cursor from other file
" complex diff parts of same file [N]
:1,2yank a | 7,8yank b
:tabedit | put a | vnew | put b
:windo diffthis 
----------------------------------------
" Vim traps
In regular expressions you must backslash + (match 1 or more)
In regular expressions you must backslash | (or)
In regular expressions you must backslash ( (group)
In regular expressions you must backslash { (count)
/fred\+/                   : matches fred/freddy but not free
/\(fred\)\{2,3}/           : note what you have to break
----------------------------------------
" \v or very magic (usually) reduces backslashing
/codes\(\n\|\s\)*where  : normal regexp
/\vcodes(\n|\s)*where   : very magic
----------------------------------------
" pulling objects onto command/search line (SUPER)
<C-R><C-W> : pull word under the cursor into a command line or search
<C-R><C-A> : pull WORD under the cursor into a command line or search
<C-R>-                  : pull small register (also insert mode)
<C-R>[0-9a-z]           : pull named registers (also insert mode)
<C-R>%                  : pull file name (also #) (also insert mode)
<C-R>=somevar           : pull contents of a variable (eg :let sray="ray[0-9]")
----------------------------------------
" List your Registers
:reg             : display contents of all registers
:reg a           : display content of register a
:reg 12a         : display content of registers 1,2 & a [N]
"5p              : retrieve 5th "ring" 
"1p....          : retrieve numeric registers one by one
:let @y='yy@"'   : pre-loading registers (put in .vimrc)
qqq              : empty register "q"
qaq              : empty register "a"
:reg .-/%:*"     : the seven special registers [N]
:reg 0           : what you last yanked, not affected by a delete [N]
"_dd             : Delete to blackhole register "_ , don't affect any register [N]
----------------------------------------
" manipulating registers
:let @a=@_              : clear register a
:let @a=""              : clear register a
:let @a=@"              : Save unnamed register [N]
:let @*=@a              : copy register a to paste buffer
:let @*=@:              : copy last command to paste buffer
:let @*=@/              : copy last search to paste buffer
:let @*=@%              : copy current filename to paste buffer
----------------------------------------
" help for help (USE TAB)
:h quickref             : VIM Quick Reference Sheet (ultra)
:h tips                 : Vim's own Tips Help
:h visual<C-D><tab>     : obtain  list of all visual help topics
                        : Then use tab to step thru them
:h ctrl<C-D>            : list help of all control keys
:helpg uganda           : grep HELP Files use :cn, :cp to find next
:helpgrep edit.*director: grep help using regexp
:h :r                   : help for :ex command
:h CTRL-R               : normal mode
:h /\r                  : what's \r in a regexp (matches a <CR>)
:h \\zs                 : double up backslash to find \zs in help
:h i_CTRL-R             : help for say <C-R> in insert mode
:h c_CTRL-R             : help for say <C-R> in command mode
:h v_CTRL-V             : visual mode
:h tutor                : VIM Tutor
<C-]>                   : jump to {keyword} under  cursor in help file [C]
<C-[>, <C-T>            : Move back & Forth in HELP History
gvim -h                 : VIM Command Line Help
:cabbrev h tab help     : open help in a tab [N]
----------------------------------------
" where was an option set
:scriptnames            : list all plugins, _vimrcs loaded (super)
:verbose set history?   : reveals value of history and where set
:function               : list functions
:func SearchCompl       : List particular function
----------------------------------------
" making your own VIM help
:helptags /vim/vim64/doc  : rebuild all *.txt help files in /doc
:help add-local-help
" save this page as a VIM Help File [N]
:sav! $VIMRUNTIME/doc/vimtips.txt|:1,/^__BEGIN__/d|:/^__END__/,$d|:w!|:helptags $VIMRUNTIME/doc
----------------------------------------
" running file thru an external program (eg php)
map   <f9>   :w<CR>:!c:/php/php.exe %<CR>
map   <f2>   :w<CR>:!perl -c %<CR>
----------------------------------------
" capturing output of current script in a separate buffer
:new | r!perl #                   : opens new buffer,read other buffer
:new! x.out | r!perl #            : same with named file
:new+read!ls
----------------------------------------
" create a new buffer, paste a register "q" into it, then sort new buffer
:new +put q|%!sort
----------------------------------------
" Inserting DOS Carriage Returns
:%s/$/\<C-V><C-M>&/g          :  that's what you type
:%s/$/\<C-Q><C-M>&/g          :  for Win32
:%s/$/\^M&/g                  :  what you'll see where ^M is ONE character
----------------------------------------
" automatically delete trailing Dos-returns,whitespace
autocmd BufRead * silent! %s/[\r \t]\+$//
autocmd BufEnter *.php :%s/[ \t\r]\+$//e
----------------------------------------
" perform an action on a particular file or file type
autocmd VimEnter c:/intranet/note011.txt normal! ggVGg?
autocmd FileType *.pl exec('set fileformats=unix')
----------------------------------------
" Retrieving last command line command for copy & pasting into text
i<c-r>:
" Retrieving last Search Command for copy & pasting into text
i<c-r>/
----------------------------------------
" more completions
<C-X><C-F>                        :insert name of a file in current directory
----------------------------------------
" Substituting a Visual area
" select visual area as usual (:h visual) then type :s/Emacs/Vim/ etc
:'<,'>s/Emacs/Vim/g               : REMEMBER you dont type the '<.'>
gv                                : Re-select the previous visual area (ULTRA)
----------------------------------------
" inserting line number into file
:g/^/exec "s/^/".strpart(line(".")."    ", 0, 4)
:%s/^/\=strpart(line(".")."     ", 0, 5)
:%s/^/\=line('.'). ' '
----------------------------------------
" *numbering lines VIM way*
:set number                       : show line numbers
:map <F12> :set number!<CR>       : Show linenumbers flip-flop
:%s/^/\=strpart(line('.')."        ",0,&ts)
" numbering lines (need Perl on PC) starting from arbitrary number
:'a,'b!perl -pne 'BEGIN{$a=223} substr($_,2,0)=$a++'
" Produce a list of numbers
" Type in number on line say 223 in an empty file
qqmnYP`n^Aq                       : in recording q repeat with @q
" increment existing numbers to end of file (type <c-a> as 5 characters)
:.,$g/^\d/exe "normal! \<c-a>"
" advanced incrementing
http://vim.sourceforge.net/tip_view.php?tip_id=150
----------------------------------------
" *advanced incrementing* (really useful)
" put following in _vimrc
let g:I=0
function! INC(increment)
let g:I =g:I + a:increment
return g:I
endfunction
" eg create list starting from 223 incrementing by 5 between markers a,b
:let I=223
:'a,'bs/^/\=INC(5)/
" create a map for INC
cab viminc :let I=223 \| 'a,'bs/$/\=INC(5)/
----------------------------------------
" *generate a list of numbers*  23-64
o23<ESC>qqYp<C-A>q40@q
----------------------------------------
" editing/moving within current insert (Really useful)
<C-U>                             : delete all entered
<C-W>                             : delete last word
<HOME><END>                       : beginning/end of line
<C-LEFTARROW><C-RIGHTARROW>       : jump one word backwards/forwards
<C-X><C-E>,<C-X><C-Y>             : scroll while staying put in insert
----------------------------------------
#encryption (use with care: DON'T FORGET your KEY)
:X                                : you will be prompted for a key
:h :X
----------------------------------------
" modeline (make a file readonly etc) must be in first/last 5 lines
// vim:noai:ts=2:sw=4:readonly:
" vim:ft=html:                    : says use HTML Syntax highlighting
:h modeline
----------------------------------------
" Creating your own GUI Toolbar entry
amenu  Modeline.Insert\ a\ VIM\ modeline <Esc><Esc>ggOvim:ff=unix ts=4 ss=4<CR>vim60:fdm=marker<esc>gg
----------------------------------------
" A function to save word under cursor to a file
function! SaveWord()
   normal yiw
   exe ':!echo '.@0.' >> word.txt'
endfunction
map ,p :call SaveWord()
----------------------------------------
" function to delete duplicate lines
function! Del()
 if getline(".") == getline(line(".") - 1)
   norm dd
 endif
endfunction

:g/^/ call Del()
----------------------------------------
" Digraphs (non alpha-numerics)
:digraphs                         : display table
:h dig                            : help
i<C-K>e'                          : enters é
i<C-V>233                         : enters é (Unix)
i<C-Q>233                         : enters é (Win32)
ga                                : View hex value of any character
#Deleting non-ascii characters (some invisible)
:%s/[\x00-\x1f\x80-\xff]/ /g      : type this as you see it
:%s/[<C-V>128-<C-V>255]//gi       : where you have to type the Control-V
:%s/[€-ÿ]//gi                     : Should see a black square & a dotted y
:%s/[<C-V>128-<C-V>255<C-V>01-<C-V>31]//gi : All pesky non-asciis
:exec "norm /[\x00-\x1f\x80-\xff]/"        : same thing
#Pull a non-ascii character onto search bar
yl/<C-R>"                         :
/[^a-zA-Z0-9_[:space:][:punct:]]  : search for all non-ascii
----------------------------------------
" All file completions grouped (for example main_c.c)
:e main_<tab>                     : tab completes
gf                                : open file under cursor  (normal)
main_<C-X><C-F>                   : include NAME of file in text (insert mode)
----------------------------------------
" Complex Vim
" swap two words
:%s/\<\(on\|off\)\>/\=strpart("offon", 3 * ("off" == submatch(0)), 3)/g
" swap two words
:vnoremap <C-X> <Esc>`.``gvP``P
" Swap word with next word
nmap <silent> gw    "_yiw:s/\(\%#\w\+\)\(\_W\+\)\(\w\+\)/\3\2\1/<cr><c-o><c-l> [N]
----------------------------------------
" Convert Text File to HTML
:runtime! syntax/2html.vim        : convert txt to html
:h 2html
----------------------------------------
" VIM has internal grep
:grep some_keyword *.c            : get list of all c-files containing keyword
:cn                               : go to next occurrence
----------------------------------------
" Force Syntax coloring for a file that has no extension .pl
:set syntax=perl
" Remove syntax coloring (useful for all sorts of reasons)
:set syntax off
" change coloring scheme (any file in ~vim/vim??/colors)
:colorscheme blue
:colorscheme morning     : good fallback colorscheme *N*
" Force HTML Syntax highlighting by using a modeline
# vim:ft=html:
" Force syntax automatically (for a file with non-standard extension)
au BufRead,BufNewFile */Content.IE?/* setfiletype html
----------------------------------------
:set noma (non modifiable)        : Prevents modifications
:set ro (Read Only)               : Protect a file from unintentional writes
----------------------------------------
" Sessions (Open a set of files)
gvim file1.c file2.c lib/lib.h lib/lib2.h : load files for "session"
:mksession                        : Make a Session file (default Session.vim)
:mksession MySession.vim          : Make a Session file named file [C]
:q
gvim -S                           : Reload all files (loads Session.vim) [C]
gvim -S MySession.vim             : Reload all files from named session [C]
----------------------------------------
#tags (jumping to subroutines/functions)
taglist.vim                       : popular plugin
:Tlist                            : display Tags (list of functions)
<C-]>                             : jump to function under cursor
----------------------------------------
" columnise a csv file for display only as may crop wide columns
:let width = 20
:let fill=' ' | while strlen(fill) < width | let fill=fill.fill | endwhile
:%s/\([^;]*\);\=/\=strpart(submatch(1).fill, 0, width)/ge
:%s/\s\+$//ge
" Highlight a particular csv column (put in .vimrc)
function! CSVH(x)
    execute 'match Keyword /^\([^,]*,\)\{'.a:x.'}\zs[^,]*/'
    execute 'normal ^'.a:x.'f,'
endfunction
command! -nargs=1 Csv :call CSVH(<args>)
" call with
:Csv 5                             : highlight fifth column
----------------------------------------
zf1G      : fold everything before this line [N]
" folding : hide sections to allow easier comparisons
zf}                               : fold paragraph using motion
v}zf                              : fold paragraph using visual
zf'a                              : fold to mark
zo                                : open fold
zc                                : re-close fold
" also visualise a section of code then type zf [N]
:help folding
zfG      : fold everything after this line [N]
----------------------------------------
" displaying "non-asciis"
:set list
:h listchars
----------------------------------------
" How to paste "normal vim commands" w/o entering insert mode
:norm qqy$jq
----------------------------------------
" manipulating file names
:h filename-modifiers             : help
:w %                              : write to current file name
:w %:r.cfm                        : change file extention to .cfm
:!echo %:p                        : full path & file name
:!echo %:p:h                      : full path only
:!echo %:t                        : filename only
:reg %                            : display filename
<C-R>%                            : insert filename (insert mode)
"%p                               : insert filename (normal mode)
/<C-R>%                           : Search for file name in text
----------------------------------------
" delete without destroying default buffer contents
"_d                               : what you've ALWAYS wanted
"_dw                              : eg delete word (use blackhole)
----------------------------------------
" pull full path name into paste buffer for attachment to email etc
nnoremap <F2> :let @*=expand("%:p")<cr> :unix
nnoremap <F2> :let @*=substitute(expand("%:p"), "/", "\\", "g")<cr> :win32
----------------------------------------
" Simple Shell script to rename files w/o leaving vim
$ vim
:r! ls *.c
:%s/\(.*\).c/mv & \1.bla
:w !sh
:q!
----------------------------------------
" count words/lines in a text file
g<C-G>                                 # counts words
:echo line("'b")-line("'a")            # count lines between markers a and b [N]
:'a,'bs/^//n                           # count lines between markers a and b
:'a,'bs/somestring//gn                 # count occurences of a string
----------------------------------------
" example of setting your own highlighting
:syn match DoubleSpace "  "
:hi def DoubleSpace guibg=#e0e0e0
----------------------------------------
" reproduce previous line word by word
imap ]  @@@<ESC>hhkyWjl?@@@<CR>P/@@@<CR>3s
nmap ] i@@@<ESC>hhkyWjl?@@@<CR>P/@@@<CR>3s
" Programming keys depending on file type
:autocmd bufenter *.tex map <F1> :!latex %<CR>
:autocmd bufenter *.tex map <F2> :!xdvi -hush %<.dvi&<CR>
" allow yanking of php variables with their dollar [N]
:autocmd bufenter *.php :set iskeyword+=\$ 
----------------------------------------
" reading Ms-Word documents, requires antiword (not docx)
:autocmd BufReadPre *.doc set ro
:autocmd BufReadPre *.doc set hlsearch!
:autocmd BufReadPost *.doc %!antiword "%"
----------------------------------------
" a folding method
vim: filetype=help foldmethod=marker foldmarker=<<<,>>>
A really big section closed with a tag <<< 
--- remember folds can be nested --- 
Closing tag >>> 
----------------------------------------
" Return to last edit position (You want this!) [N]
autocmd BufReadPost *
     \ if line("'\"") > 0 && line("'\"") <= line("$") |
     \   exe "normal! g`\"" |
     \ endif
----------------------------------------
" store text that is to be changed or deleted in register a
"act<                                 :  Change Till < [N]
----------------------------------------
"installing/getting latest version of vim on Linux (replace tiny-vim) [N]
yum install vim-common vim-enhanced vim-minimal
----------------------------------------
# using gVIM with Cygwin on a Windows PC
if has('win32')
source $VIMRUNTIME/mswin.vim
behave mswin
set shell=c:\\cygwin\\bin\\bash.exe shellcmdflag=-c shellxquote=\"
endif
----------------------------------------
" *Just Another Vim Hacker JAVH*
vim -c ":%s%s*%Cyrnfr)fcbafbe[Oenz(Zbbyranne%|:%s)[[()])-)Ig|norm Vg?"
----------------------------------------
vim:tw=78:ts=8:ft=help:norl:
__END__
----------------------------------------
"Read Vimtips into a new vim buffer (needs w3m.sourceforge.net)
:tabe | :r ! w3m -dump http://zzapper.co.uk/vimtips.html    [N]
----------------------------------------
updated version at http://www.zzapper.co.uk/vimtips.html
----------------------------------------
Please email any errors, tips etc to
vim@rayninfo.co.uk
" Information Sources
----------------------------------------
www.vim.org
Vim Wiki *** VERY GOOD *** [N]
Vim Use VIM newsgroup [N]
comp.editors
groups.yahoo.com/group/vim "VIM" specific newsgroup
VIM Webring
VimTips PDF Version (PRINTABLE!)
Vimtips in Belarusian 
----------------------------------------
" : commands to neutralise < for HTML display and publish
" use yy@" to execute following commands
:w!|sav! vimtips.html|:/^__BEGIN__/,/^__END__/s#<#\<#g|:w!|:!vimtipsftp
----------------------------------------
\end{verbatim}
\end{multicols}
\end{document}
